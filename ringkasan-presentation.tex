%\documentclass{beamer}

%% - begin - for create handout
\documentclass[handout]{beamer}
\usepackage{pgfpages}
%\pgfpagesuselayout{2 on 1}[a4paper, border shrink=5mm] % potrait
\pgfpagesuselayout{4 on 1}[a4paper, border shrink=5mm, landscape]
%% - end - for create handout

%\usetheme{Boadilla}
%\usetheme{Bergen}
%\usetheme{Madrid}
%\usetheme{Antibes}
%\usetheme{Hannover}
%\usetheme{Singapore}
\usetheme{Warsaw}

%\usecolortheme{crane}



\title{Contoh Presentasi dengan Latex}
\subtitle{Pake Beamer}
\author{tamami}
\institute{BPPKAD Kab. Brebes}
\date{\today}

\begin{document}

%%% - frame 1

\begin{frame}
\titlepage
\end{frame}


%%% - frame 2

\begin{frame}
\section{Ini bagian pertama}
\subsection{Ini subbagian pertama}
\end{frame}

%%% - frame 3

\begin{frame}
\frametitle{Contoh Halaman Pertama}
Ini isinya
\end{frame}


%%% - frame 4

\begin{frame}
\frametitle{Outline}
\tableofcontents
\end{frame}


%%% - frame 5

\begin{frame}
\frametitle{List}
\begin{itemize}
	\item Point A
	\item Point B
	\begin{itemize}
		\item Part 1
		\item Part 2
	\end{itemize}
\end{itemize}
\end{frame}


%%% - frame 6

\begin{frame}
\frametitle{Penomoran List}
\begin{enumerate}
	\item Bagian 1
	\item Bagian 2
	\begin{enumerate}[a.]
		\item poin a
		\item poin b
	\end{enumerate}
\end{enumerate}
\end{frame}


%%% - frame 7

\begin{frame}
\frametitle{Kolom}
\begin{columns}
\column{0.5\textwidth}
ini kolom pertama, yang seharusnya bisa penuh isinya
\column{0.5\textwidth}
ini kolom kedua yang seharusnya bisa penuh juga isinya, tapi sepertinya juga ga begitu kepake siy di slide presentasi.. siapa juga yang mau baca paragraf panjang di sebuah presentasi kalo bukan pas lagi bawaan puisi.
\end{columns}
\end{frame}


%%% - frame 8

\begin{frame}
\frametitle{Gambar}
\begin{figure}
	\includegraphics[scale=0.25]{lion}
	\caption{lion!!!}
\end{figure}
dibawahnya bisa dikasih teks penjelasan.
\end{frame}


%%% - frame 9

\begin{frame}
\frametitle{Deskripsi}
\begin{description}
\item [API] Application Programming Interface
\item [LAN] Local Area Network
\item [ASCII] American Standard Code for Information Interchange
\end{description}
\end{frame}


%%% - frame 10

\begin{frame}
\frametitle{Tabel}
\begin{table}
	\begin{tabular}{l | c | l }
		NIM & Nama & Tgl Lahir \\
		\hline \hline
		17001 & tamami & 9 April 1984 \\
		17002 & yanto & 15 Januari 1985 \\
		17003 & ami & 30 September 1984 
	\end{tabular}
	\caption{Data Mahasiswa}
\end{table}
\end{frame}


%%% - frame 11

\begin{frame}
\frametitle{Blok}
\begin{block}{Blok Biasa}
Ini isi blok biasa
\end{block}

\begin{alertblock}{Ini Alert Blok}
Ini isi alert blok
\end{alertblock}

\begin{definition}
Ini isi blok definisi
\end{definition}

\begin{example}
Ini isi blok contoh
\end{example}
\end{frame}


%%% - frame 12

\begin{frame}
\frametitle{Blok lainnya}
\begin{theorem}[Pythagoras]
$ a^2 + b^2 = c^2 $
\end{theorem}

\begin{corollary}
$ x + y = y + x $
\end{corollary}

\begin{proof}
$ \omega + \phi = \epsilon $
\end{proof}
\end{frame}


%%% - frame 13

\begin{frame}[fragile]
\frametitle{Kode}
\begin{semiverbatim}
	\\begin\{frame\}
	\\frametitle\{Outline\}
	\\tableofcontents
	\\end\{frame\}
\end{semiverbatim}
\end{frame}


%%% - frame 14

\begin{frame}
\frametitle{Hyperlink}
\label{contens}
\hyperlink{contents}{klik disini}

\hyperlink{contents}{\beamerbutton{contents page}}

\hyperlink{contents}{\beamergotobutton{columns page}}

\hyperlink{contents}{\beamerskipbutton{pictures page}}

\hyperlink{contents}{\beamerreturnbutton{pictures page}}
\end{frame}


%%% - frame 15 - atur timing keluarnya teks

\begin{frame}
\frametitle{Timing List}
	\begin{itemize}
		\pause
		\item Poin A
		\pause
		\item Poin B
		\begin{itemize}
			\pause
			\item bagian 1
			\pause
			\item bagian 2
		\end{itemize}
		\pause
		\item Poin C
	\end{itemize}
\end{frame}


%%% - frame 16 - another timing list with overlay spec

\begin{frame}
\frametitle{Another Timing List}
\begin{enumerate}[(I)]
	\item<1-> Poin A
	\item<2-> Poin B
	\begin{itemize}
		\item<3-> bagian 1
		\item<4-> bagian 2
	\end{itemize}
	\item<5-> Poin C
\end{enumerate}
\end{frame}


%%% - frame 17 - another flipping with timing

\begin{frame}
\frametitle{Flipping}
\onslide<1,3,5-6>{Teks pertama}

\onslide<2,4,6>{Teks kedua}
\end{frame}


%%% - frame 18 - transparent in timing list

\begin{frame}
\frametitle{Transparent List}
\setbeamercovered{transparent}
\begin{itemize}
	\item<1-> Poin A
	\item<2-> Poin B
	\item<3-> Poin C
\end{itemize}
\end{frame}


%%% - frame 19 - text formatting

\begin{frame}
\frametitle{Format Text}
\textbf<2>{Contoh Teks}\\
\textit<2>{Contoh Teks}\\
\textsl<2>{Contoh Teks}\\
\textrm<2>{Contoh Teks}\\
\textsf<2>{Contoh Teks}\\
\textcolor<2>{orange}{Contoh Teks}\\
\alert<2>{Contoh Teks}\\
\structure<2>{Contoh Teks}
\end{frame}


%%% - frame 20 - env overlay

\begin{frame}
\frametitle{Rumus}
\begin{theorem}<1->[Pythagoras]
$ a^2 + b^2 = c^2 $
\end{theorem}

\begin{proof}<2->
$ \omega + \phi = \epsilon $
\end{proof}
\end{frame}


%%% - frame 21 - table overlay

\begin{frame}
\frametitle{Tabel Overlay}
\begin{table}
\begin{tabular}{r | l | c}
	NIM & Nama & Gaji \\
	\hline \hline
	17001 & tamami & 5.000.000 \onslide<2-> \\
	17002 & yanto & 5.500.000 \onslide<3-> \\
\end{tabular}
\caption{Data Pegawai}
\end{table}
\end{frame}

\end{document}